\par Além do que foi dado no enunciado, um dado grafo $C_{2n} = (V, E)$ só é um
circuito $2n$ se existem apenas $2$ caminhos de um vértice inicial $v$ para um
oposto $u$ e ambos os caminhos têm distância $dist(v, u) = n$. Então, no caso para
um subgrafo $C_4$, podemos testar todos os caminhos de distância $2$, marcando cada
um que for percorrido. Se for o segundo caminho achado, então existe um subgrafo $C_4$.

\par Nesse caso, a complexidade para decidir se, de um dado vértice $v$,
encontramos um $C_4$, sendo $d(v)$ o número de arestas de um vértice, é:
\begin{equation*}
    O\left(\sum_W d(w)\right)\text{; com}\ W = \{w:\ \forall w \in V \mid w\
    \text{é adjacente a}\ v\}
\end{equation*}
\par Que no pior caso é:
\begin{align*}
    O\left(\sum_W d(w)\right)
    &= O\left(\sum_V |V|\right) \\
    &= O\left(|V| \times |V|\right) \\
    &= O\left(|V|^2\right)
\end{align*}

\par Testando isso para todos os vértices fica, então, $|V| \times O\left(|V|^2
\right) = O\left(|V|^3\right)$. Além disso, o algoritmo abaixo poderia ser melhorado
evitando as arestas já testadas, porém a complexidade continuaria a mesma.